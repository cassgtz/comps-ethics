\documentclass[10pt,twocolumn]{article} 

% required packages for Oxy Comps style
\usepackage{oxycomps} % the main oxycomps style file
\usepackage{times} % use Times as the default font
\usepackage[style=numeric,sorting=nyt]{biblatex} % format the bibliography nicely

\usepackage{amsfonts} % provides many math symbols/fonts
\usepackage{listings} % provides the lstlisting environment
\usepackage{amssymb} % provides many math symbols/fonts
\usepackage{graphicx} % allows insertion of grpahics
\usepackage{hyperref} % creates links within the page and to URLs
\usepackage{url} % formats URLs properly
\usepackage{verbatim} % provides the comment environment
\usepackage{xpatch} % used to patch \textcite

\bibliography{references}
\DeclareNameAlias{default}{last-first}

\xpatchbibmacro{textcite}
  {\printnames{labelname}}
  {\printnames{labelname} (\printfield{year})}
  {}
  {}

\pdfinfo{
    /Title (CS Comps Literature Review)
    /Author (Cassandra Gutierrez)
}

\title{CS Comps Literature Review}

\author{Cassandra Gutierrez}
\affiliation{Occidental College}
\email{gutierrezc@oxy.edu}

\begin{document}

\maketitle

\begin{abstract}
This paper serves as an initial literature review for my project proposal, a note taking app that supports one-way, real-time collaboration. Specifically, this paper will define the problem that the project would solve, explain the technical background needed to understand the project, and describe existing solutions to similar problems.
\end{abstract}

\section{Problem Context}
My project is an attempt to solve the issue of cognitive overload in students with regard to note-taking during math lectures. Although it has been a long-standing practice, recent research suggests that traditional note-taking is not an effective approach to learning. For instance, the book Efficiency in Learning states, “The cognitive effort required to take notes reduces mental capacity that could be devoted to processing the content in ways that lead to learning.”\cite{clark_nguyen_sweller_baddeley_2006} In other words, note-taking requires a split cognitive effort that takes away from students’ efforts and abilities to fully comprehend the material from a lecture. Furthermore, a research paper in education psychology found that note-taking under the constraints of listening results in cognitive overload that reduces students’ recall abilities.\cite{rickards_fajen_sullivan_gillespie_1997} This means that if students had the ability to devote their full attention to a lecture it would enhance their ability to process, understand, and retain material. 

In addition to supporting research, I also have personally found it difficult to keep up with professors during lectures throughout my college career. It is most prevalent in mathematics courses, where most professors are fast-paced and expect students to copy down long solutions as they explain them. At times I find it pointless to go to lectures because I don’t learn anything. I’m too occupied copying things down to be able to process what the professor is saying. The professor’s voice is just background noise at that point. For this reason, I’d like to tailor this project to solve cognitive overload in note-taking for math lectures. 

A note taking app that supports one-way, real-time collaboration would solve the need for students to constantly copy down lecture notes, allowing them to devote more attention to processing and understanding the material. The idea is simple. A professor would have a source file that contains the lecture notes. Every student would have their own copy (target file) of the source file. As the professor adds notes or makes edits to the source file, all target files will receive those changes in real-time. In addition, every student can make edits to their individual target file in order to customize their notes. However, students’ edits would not affect any other target file or the source file. This would relieve students from the cognitive overload of copying down notes because they are receiving them from the professor in real-time. In turn, it would allow students to devote more attention to listening to the lecture, comprehending, and processing information.

\section{Technical Background}
This project requires a one-way synchronization algorithm in order to achieve real-time concurrency. Synchronization algorithms are designed to sync changes in files in real-time, handling conflict resolution and updating the current state of the file, so that changes can be instantaneously reflected to all files. One-way synchronization involves reflecting changes only from the source files to target files and not the other way around. This means any changes made to target files will not affect the source file or any other target file. Although there is no existing algorithm that implements one-way synchronization, there are existing algorithms that solve two-way synchronization which can be useful to draw from when designing a one-way synchronization algorithm. 

Two-way synchronization involves reflecting changes to all files, regardless if they were made to a source or target file. In other words, if changes are made to a source file they will be reflected in all target files and if changes are made to a target file they will be reflected in the source file and all other target files. The main two-way, real-time synchronization algorithms used for collaborative editing are Operational Transformation (OT) and Differential Synchronization (DS). 
\begin{itemize}
    \item \textbf{Operational Transformation:} OT is the most common algorithm used for real-time collaborative editors today. It works by capturing every single change of characters (operations) in the form “insert character ‘Y’ at index 1, delete character ‘X’ at index 0.”~\cite{gala_2019} Every operation is sent to a server, which then applies and broadcasts them to every other file in the network. This technique is fitting because it can function across a network with high-latency. In this case, a college campus internet server. 
    \item \textbf{Differential Synchronization:} DS is a symmetrical algorithm that utilizes a never-ending cycle of background operations called difference (diff) and patch.~\cite{35605} Diff is used to get the differences (or changes) between target files and source files and store them all in a patch file. Patch is then used to apply the differences to the target files, converting them into an identical copy of the source file. This technique is fitting because it can handle text \textit{and} bitmaps, which would support a sketch style document that is usually utilized in math lectures. 
\end{itemize}

While these algorithms are known to be difficult to implement, there are basic open source implementations of them available on the internet. Exploring and drawing from their implementations would aid in creating a solution to a one-way synchronization algorithm.

\section{Prior Work}
There are dozens of note taking apps already on the market. For instance, Notion, Evernote, Google Docs, ect. This project would draw most of the user interface aspects and editing tools from the existing note taking apps. For example, the typical tools bar at the top of the page with options to change drawing colors, edit size point of pen, an eraser, and so on. 

The real distinction between this project and other note apps would be the real-time, one-way aspect of collaborative editing. Google Docs does provide real-time, two-way collaboration on the same document through Operational Transformation. Nevertheless, there is no possible instance in which one user can make an edit to a document and other users would not receive those edits in the same document. A solution could be for multiple users to download or make their own copy of a document and apply their edits in their separate copy. However, in this problem context, that would require pre-made notes from professors, which is usually not the case in math courses. Professors typically start off with a blank page/board and work through problems in real time. For these reasons, a real-time, one-way solution is necessary and distinct. 

\printbibliography
\end{document}